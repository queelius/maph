\PassOptionsToPackage{unicode}{hyperref}
\PassOptionsToPackage{naturalnames}{hyperref}

\definecolor{ltgray}{rgb}{0.7,0.7,0.7}

\lstset
{
    basicstyle=\footnotesize,               % print whole listing small
    keywordstyle=\bfseries,
    identifierstyle=,               % nothing happens
    commentstyle=\itshape\color{ltgray},    % gray comments
    %stringstyle=\ttfamily,          % typewriter type for strings
    showstringspaces=false,         % no special string spaces
    numbers=left,
    numberstyle=\tiny,
    stepnumber=1,
    numbersep=5pt
}

\hypersetup{
    pdftitle={The \emph{Perfect Hash Filter}},  % title
    pdfauthor={Alexander Towell},               % author
    pdfsubject={computer science},              % subject of the %document
    pdfkeywords={
        probabilistic data structure,
        abstract data type,
        approximate set,
        bloom filter,
        perfect hash function,
        perfect hash filter},                 % keywords
    colorlinks=true,                           % false: boxed links;
    linkcolor=blue,
    citecolor=green,                            % color of links to 
    filecolor=magenta,                          % color of file links
    urlcolor=green                              % color of external
}

\newcommand{\catop}{{+\!\!+}}
\newcommand{\cisb}{\mathbb{B}}
\newcommand{\rv}[1]{\mathrm{#1}}
\newcommand{\pmf}{p}
\newcommand{\cdf}{F}
\newcommand{\prob}{\Pr} %\mathrm{Pr}
\newcommand{\ith}{$i^{\text{th }}$}
\newcommand{\setc}[1]{\overline{#1}}

\newcommand{\dudist}{\operatorname{\rm{DU}}}
\newcommand{\berdist}{\operatorname{\rm{BER}}}
\newcommand{\expdist}{\operatorname{\rm{EXP}}}
\newcommand{\geodist}{\operatorname{\rm{GEO}}}
\newcommand{\card}[1]{\left\vert{#1}\right\vert}
\newcommand{\given}{{\,|\,}}
\newcommand{\indicator}[1]{\operatorname{\mathbbm{1}_{#1}}}
\renewcommand{\vec}[1]{\boldsymbol{\mathbf{#1}}}
\newcommand{\matrx}[1]{\boldsymbol{#1}}
\newcommand*{\euler}{\mathrm{e}}
\newcommand{\St}{\mathbb{S}}
\newcommand{\Sa}{\St^*}
\newcommand{\Sp}{\St^+}
\newcommand{\Sn}{\St^-}
\newcommand{\Os}{\check{\St}^*}
\newcommand{\Osp}{\check{\St}^+}
\newcommand{\Osn}{\check{\St}^-}


\DeclareMathOperator{\likeli}{\mathcal{L}}
\DeclareMathOperator{\expectation}{\mathbb{E}}
\DeclareMathOperator*{\argmax}{arg\,max}
\DeclareMathOperator*{\argmin}{arg\,min}

\SetKwFunction{AE}{e$^*\!$}
\SetKwFunction{RE}{e}

\newcommand*{\defeq}{\stackrel{\text{def}}{=}}

\SetKwFunction{hash}{h}
\SetKwFunction{ph}{ph}
\SetKwFunction{bf}{BF}
\SetKwFunction{Contains}{contains}
\SetKwFunction{MakeObliviousSet}{make\_oblivious\_set}
\SetKwFunction{MakeApproxSet}{make\_perfect\_hash\_filter}
\SetKwFunction{MakeTypeApproxSet}{make\_typed\_perfect\_hash\_filter}
\SetKwFunction{MakeKPerfectHashSet}{make\_k\_perfect\_hash\_filter}

\SetKwFunction{Find}{find}
\SetKwFunction{Encode}{encode}
\SetKwFunction{Decode}{decode}
\SetKwFunction{ro}{h$^*$}
\SetKwFunction{PHFgen}{make\_perfect\_hash}
\SetKwFunction{very}{Very}
\SetKwFunction{BL}{BL}
\SetKwFunction{somewhat}{Somewhat}
\SetKwFunction{Rows}{rows}
\SetKwFunction{Columns}{columns}
\SetKwFunction{Count}{count}
\SetKwFunction{Cardinality}{cardinality}
\SetKwFunction{FalsePositiveRate}{false\_positive\_rate}
\SetKwFunction{Union}{union}
\SetKwFunction{Complement}{complement}
\SetKwFunction{Intersection}{intersection}
\SetKwFunction{concat}{concatenate}
\SetKwFunction{Powerset}{powerset}
\SetKwFunction{sampler}{bit\_length\_sampler}

\SetKwFunction{FalseNegativeRate}{false\_negative\_rate}
\SetKwFunction{Union}{union}
\SetKwFunction{Complement}{complement}
\SetKwFunction{Intersection}{intersection}
\SetKwFunction{Difference}{difference}

\SetKw{True}{true}
\SetKw{False}{false}
\SetKw{Bool}{Boolean}
\SetKw{Break}{break}
\SetKw{PHF}{PHF}
\SetKw{MPHF}{MPHF}

\SetKwData{found}{found}

\SetKwInOut{Params}{params}
\SetKwInOut{KwIn}{in}
\SetKwInOut{KwOut}{out}    

\RestyleAlgo{ruled}
\LinesNumbered

\newcounter{equationstore}
\AtBeginEnvironment{proof}{\setcounter{equationstore}{\value{equation}}
\setcounter{equation}{0}\renewcommand{\theequation}{\alph{equation}}}
\AtEndEnvironment{proof}{\setcounter{equation}{\value{equationstore}}}

\newcounter{examplecounter}
\newenvironment{example}
{
    \begin{quote}%
    \refstepcounter{examplecounter}%
    \textbf{Example \arabic{examplecounter}}%
    \quad
}
{%
    \end{quote}%
}

\newcounter{equationex}
\AtBeginEnvironment{example}{\setcounter{equationex}{\value{equation}}
\setcounter{equation}{0}\renewcommand{\theequation}{\alph{equation}}}
\AtEndEnvironment{example}{\setcounter{equation}{\value{equationex}}}



\theoremstyle{plain}
\newtheorem{theorem}{Theorem}
\newtheorem{corollary}{Corollary}
\newtheorem{definition}{Definition}
\newtheorem{postulate}{Postulate}
\newtheorem{assumption}{Assumption}
\newtheorem{conjecture}{Conjecture}

\theoremstyle{remark}
\newtheorem*{remark}{Remark}
\newtheorem*{notation}{Notation}

\setglossarystyle{list}
\glsdisablehyper
\makeglossaries

\bibliographystyle{plainnat}

\usetikzlibrary{arrows,shapes,positioning}
\usetikzlibrary{calc,decorations.markings}

\graphicspath{{img/}} 

\crefformat{chapter}{\S#2#1#3}
\crefmultiformat{chapter}{\S\S#2#1#3}{and~#2#1#3}{, #2#1#3}{, and~#2#1#3}

\crefformat{section}{\S#2#1#3}
\crefmultiformat{section}{\S\S#2#1#3}{and~#2#1#3}{, #2#1#3}{, and~#2#1#3}