\documentclass[ ../main.tex]{subfiles}
\providecommand{\mainx}{..}
\begin{document}
\section{Applications}
The \emph{approximate map} trades \emph{accuracy} for \emph{space efficiency}. Therefore, the stored representation of the keys is lossy which makes recovering a key from its representation impractical. Thus, the \emph{approximate map} facilitates the implementation of space and time efficient non-iterable approximate containers.

\subsection{Approximate multi-set}
A multi-set (bag) is given by the following definition.
\begin{definition}
A multi-set is an unordered collection of elements from a universe where each element may occur multiple times. More formally, a multi-set $\Set{S}$ is a group of elements with the following properties:
\begin{enumerate}
    \item All elements belong to a universe $\Set{U}$.
    \item For each $x \in \Set{U}$, $x$ is either a member of $\Set{S}$ or it is not. If $x$ is a member, it may occur multiple times, denoted the multiplicity of $x$.
    \item The elements in $\Set{S}$ are not ordered.
\end{enumerate}
\end{definition}
We are interested in the countable multi-sets. A countable multi-set is a \emph{finite multi-set} or a \emph{countably infinite multi-set}. A \emph{finite multi-set} has a finite number of elements. For example,
\[
    \{ 1, 3, 1, 1 \}
\]
is a finite set with two distinct elements, where element $1$ has a multiplicity of $3$ and element $3$ has a multiplicity of $1$. A \emph{countably infinite multi-set} can be put in one-to-one correspondence with the set of natural numbers.

The cardinality of a multi-set $\Set{A}$ is a measure of the number of elements in the multi-set, denoted by
\begin{equation}
    \Card{\Set{A}}\,.
\end{equation}
The cardinality of a \emph{finite multi-set} is a non-negative integer and counts the number of elements in the set, e.g.,
\[
    \Card{\left\{ 1, 3, 1, 1\right\}} = 4\,.
\]

The \gls{gls-pmf} may implement an approximate multi-set where the multiplicy of $x$ is given by
\begin{equation}
    \Mp[x]\,.
\end{equation}
The space complexity of a multi-set implemented using the \gls{gls-pmf} is given approximately by
\begin{equation}
    m \log_2 \!\left( \frac{n}{\fprate} \right) + 1.44 m\,.
\end{equation}
where $m$ is number of unique elements in the multi-set, $n$ is the largest multiplicity of any of the elements in the multi-set, and $\fprate$ is the false positive rate.

\subsection{Mutable approximate sets}
\label{sec:mutable}
In the approximate hash map, deleting an $x \in \PASet{S} \setminus \Set{S}$ will cause a false negative on an element $y \in \Set{S}$, where $\ph(y) = \ph(x)$.

After $k-1$ insertions have succeeded, inserting an $x \in \overline{\PASet{S}}$ succeeds with probability $1-r_k$ where $r_k$ is the load factor after $k$ successful insertions and is given by
\begin{equation}
    r_k = r_{k-1}\left(1 + \frac{1}{m_{k-1}}\right)\,,
\end{equation}
where $m_k$ is the cardinality of $\Set{S}$ after $k$ successful insertions with a base case given by $r_0$.

The false positive rate after $k$ successful insertions is given by
\begin{equation}
    \fprate = r_k 2^{-M}\,.
\end{equation}

Attempting to insert an $x \in \overline{\PASet{S}}$\footnote{For any $x \in \PASet{S}$, it is already a member.} into $\Set{S}$ may fail due to colliding with an existing member.
We could remember this outcome and say that the given $x$ is not a member of $\Set{S}$.
However, we take the approach of quantifying the outcome of an insertion probabilistically.

\begin{definition}
The discrete random variable $\RV{K} \in \{0, 1, \ldots\}$ denotes the number of successful insertions.
\end{definition}

Given a $\pmapf$ The probability mass function that $\RV{K} = k$ insertions succeed given $n$ insertion attempts and $t$ deletions is given by the recurrence relation
\begin{equation}
    \PDF{k \Given m_0, n, t}[\RV{K}] = \PDF{k-1 \Given n-1}[\RV{K}]\left(1 - r_{k-1}\right) + \PDF{k \Given n-1}[\RV{K}] r_k\,,
\end{equation}


The probability mass function that $\RV{K} = k$ insertions succeed given $n$ insertion attempts is given by the recurrence relation
\begin{equation}
    \PDF{k \Given n}[\RV{K}] = \PDF{k-1 \Given n-1}[\RV{K}]\left(1 - r_{k-1}\right) + \PDF{k \Given n-1}[\RV{K}] r_k\,,
\end{equation}
with a base case given by
\begin{equation}
    \PDF{k \Given 0}[\RV{K}] = \SetIndicator{k=0}\,.
\end{equation}

Since we know the distribution of successful inserts after $n$ insertions, we may compute the false negative rate that results from this.

A false negative occurs for an element $x$ whenever we insert $x$ and the insertion fails. If we try to insert $n$ elements not in $\PASet{S}$ (note that successfully inserting an element $x$ increases the false positive rate 

See \Cref{sec:guarded} to see how to make deletions and insertions deterministically succeed over a \emph{guarded} set.



%\documentclass[ ../main.tex]{subfiles}
\providecommand{\mainx}{..}
\begin{document}
\subsection{Postings list}
\subsubsection{Boolean search}
Suppose we have a collection of sets, denoted the universe $\Set{D}$. Set-theoretic queries operate on subsets of $\Set{D}$. All set-theoretic queries are reducible to compositions of \emph{set complement} and \emph{set intersection}. Set complement is given by
\begin{equation}
    \notfn\!\left(\Set{A}\right) = \left\{\Set{Z} \in \Set{U} \colon \Set{Z} \notin \Set{A}\right\}
\end{equation}
and set intersection is given by
\begin{equation}
    \andfn\!\left(\Set{A}, \Set{B}\right) = \left\{\Set{Z} \in \Set{U} \colon \text{$\Set{Z} \in \Set{A}$ and $\Set{Z} \in \Set{B}$}\right\}\,,
\end{equation}
where $\Set{A}$ and $\Set{B}$ are subsets of $\Set{U}$. For instance, set difference $\Set{A} \setminus \Set{B}$, the elements in $\Set{A}$ and not in $\Set{B}$, is given by
\begin{equation}
    \andfn\!\left(\Set{A}, \notfn(\Set{B})\right)\,.
\end{equation}

A recursive query language may support arbitrarily complex set-theoretic queries as given by BNF~\ref{bnf:set_theoretic}.
\begin{figure}
\caption{BNF of set-theoretic query grammar}
\label{bnf:set_theoretic}
\begin{bnf*}
    \bnfprod{query}
        \bnftd{a membership test} \bnfor \bnfpn{unary exp} \bnfor \bnfpn{binary exp}\\
    \bnfprod{unary exp} \bnfpn{unary op} \bnfts{(}\bnfpn{query}\bnfts{)}\\
    \bnfprod{binary exp}
        \bnfpn{binary op} \bnfts{(}\bnfpn{query}\bnfts{,}\bnfpn{query}\bnfts{)}\\
    \bnfprod{unary op} \bnfts{$\notfn$} \bnfor \bnfsk\\
    \bnfprod{binary op} \bnfts{$\andfn$} \bnfor \bnfts{$\orfn$} \bnfor \bnfts{$\diffn$} \bnfor \bnfsk
\end{bnf*}
\end{figure}



\begin{example}
Suppose we are interested in those sets which contain element $x$ but not element $y$, or those sets which contain element $y$. We may reduce this query to the following composition. Let
\begin{equation*}
    \Set{A} = \left\{ \Set{Z} \in \Set{U} \colon \Has(\Set{Z}, x) \right\}
\end{equation*}
and
\begin{equation*}
    \Set{B} = \left\{ \Set{Z} \in \Set{U} \colon \Has(\Set{Z}, x) \right\}\,,
\end{equation*}
which are the result of simple membership tests for each set in the universe. Given these two sets, we are interested in the set
\begin{equation*}
    \orfn\!\left(\diffn\!\left(\Set{A}, \Set{B}\right), \Set{B}\right)\,.
\end{equation*}
\end{example}

\subsubsection{Rank-ordered search}
\label{sec:extensions:fuzzyset}
Fuzzy sets are a super-set of  classical (crisp) sets. In a \emph{fuzzy set}, an element has a degree-of-membership between $0$ and $1$. Classical sets are a special case of fuzzy sets where all elements either have $0$ degree-of-membership or $1$ degree-of-membership.

The \emph{approximate map} may be used to implement \emph{approximate fuzzy sets}, where the keys are set elements and the associated values encode degree-of-membership values.

Suppose the values are $p$ bits. Then, let the degree-of-membership be with respect to a function
\begin{equation}
    \operatorname{u} \colon \cisb_P \mapsto [0,1]\,,
\end{equation}
which maps the $P$ bits associated with a particular element $x \in \Sp$ to a degree-of-membership.\footnote{An element $x \notin \Sp$ has $0$ membership in $\Sp$.}
\begin{example}
Let the $p$ bits $b_1, \ldots, b_P$ encode an unsigned integer between $0$ and $2^p-1$. Then,
\begin{equation}
    \operatorname{u}\!\left(b_1, \ldots, b_p\right) = \frac{\sum_{j=1}^{P} 2^{j-1} b_j}{2^P-1}
\end{equation}
maps those $p$ bits to a rational number between $0$ and $1$.
\end{example}

Fuzzy set-theoretic queries operate on degree-of-membership values given by $\operatorname{u}$. Fuzzy operator equivalents to Boolean $\notfn$, $\orfn$, and $\andfn$ are given by
\begin{align}
    \notfn(x) &= 1 - x\,,\\
    \orfn(x, y) &= \max(x, y)\,,\\
    \andfn(x, y) &= \min(x, y)\,,
\end{align}
where $x$ and $y$ are degree-of-membership values.

A hedge function transforms degree-of-membership values as demonstrated by the following example.
\begin{example}
It may be true that an element is a member of a set $\Set{S}$ but it may not be \emph{very} true. If we are interested in sets in which it is \emph{very} true that an element is a member, then the hedge function given by
\begin{equation}
    \very(x) = x^2
\end{equation}
makes the appropriate transformation since $\true\left(\very(x)\right)$ is \emph{false} for larger values of $x$ than $\true(x)$.
\end{example}

A Boolean membership test is eventually needed, in which case defuzzification transforms degree-of-membership values into crisp $\true$ or $\false$ values. For instance,
\begin{equation}
    \true(x) = x \geq K\,,
\end{equation}
where $K \in [0,1]$, transforms degree-of-membership values larger than $K$ to $\true$ and otherwise $\false$. At this point, standard set-theoretic queries as given by BNF~\ref{bnf:set_theoretic} may be used.



In \cite{}, we discussed the fuzzy approximate set. Here, we show how the degree-of-membership may be a function of various other properties having to do with word proximity and word frequency.

Let the degree-of-membership be with respect to a function
\begin{equation}
    \operatorname{u} \colon \cisb \mapsto [0,1]\,,
\end{equation}
which maps the bits associated with a particular element $w \in \Set{W}$ to a degree-of-membership.

Fuzzy set-theoretic queries operate on degree-of-membership values given by $\operatorname{u}$. Fuzzy operator equivalents to Boolean $\notfn$, $\orfn$, and $\andfn$ are given by
\begin{align}
    \notfn(x) &= 1 - x\,,\\
    \orfn(x, y) &= \max(x, y)\,,\\
    \andfn(x, y) &= \min(x, y)\,,
\end{align}
where $x$ and $y$ are degree-of-membership values.

A hedge function transforms degree-of-membership values as demonstrated by the following example.
\begin{example}
It may be true that an element is a member of a set $\Set{S}$ but it may not be \emph{very} true. If we are interested in sets in which it is \emph{very} true that an element is a member, then the hedge function given by
\begin{equation}
    \very(x) = x^2
\end{equation}
makes the appropriate transformation since $\true\left(\very(x)\right)$ is \emph{false} for larger values of $x$ than $\true(x)$.
\end{example}

A Boolean membership test is eventually needed, in which case defuzzification transforms degree-of-membership values into crisp $\true$ or $\false$ values. For instance,
\begin{equation}
    \true(x) = x \geq K\,,
\end{equation}
where $K \in [0,1]$, transforms degree-of-membership values larger than $K$ to $\true$ and otherwise $\false$. At this point, standard set-theoretic queries as given by BNF~\ref{bnf:set_theoretic} may be used.
\end{document}
\end{document}
