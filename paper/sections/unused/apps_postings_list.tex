\documentclass[ ../main.tex]{subfiles}
\providecommand{\mainx}{..}
\begin{document}
\subsection{Postings list}
\subsubsection{Boolean search}
Suppose we have a collection of sets, denoted the universe $\Set{D}$. Set-theoretic queries operate on subsets of $\Set{D}$. All set-theoretic queries are reducible to compositions of \emph{set complement} and \emph{set intersection}. Set complement is given by
\begin{equation}
    \notfn\!\left(\Set{A}\right) = \left\{\Set{Z} \in \Set{U} \colon \Set{Z} \notin \Set{A}\right\}
\end{equation}
and set intersection is given by
\begin{equation}
    \andfn\!\left(\Set{A}, \Set{B}\right) = \left\{\Set{Z} \in \Set{U} \colon \text{$\Set{Z} \in \Set{A}$ and $\Set{Z} \in \Set{B}$}\right\}\,,
\end{equation}
where $\Set{A}$ and $\Set{B}$ are subsets of $\Set{U}$. For instance, set difference $\Set{A} \setminus \Set{B}$, the elements in $\Set{A}$ and not in $\Set{B}$, is given by
\begin{equation}
    \andfn\!\left(\Set{A}, \notfn(\Set{B})\right)\,.
\end{equation}

A recursive query language may support arbitrarily complex set-theoretic queries as given by BNF~\ref{bnf:set_theoretic}.
\begin{figure}
\caption{BNF of set-theoretic query grammar}
\label{bnf:set_theoretic}
\begin{bnf*}
    \bnfprod{query}
        \bnftd{a membership test} \bnfor \bnfpn{unary exp} \bnfor \bnfpn{binary exp}\\
    \bnfprod{unary exp} \bnfpn{unary op} \bnfts{(}\bnfpn{query}\bnfts{)}\\
    \bnfprod{binary exp}
        \bnfpn{binary op} \bnfts{(}\bnfpn{query}\bnfts{,}\bnfpn{query}\bnfts{)}\\
    \bnfprod{unary op} \bnfts{$\notfn$} \bnfor \bnfsk\\
    \bnfprod{binary op} \bnfts{$\andfn$} \bnfor \bnfts{$\orfn$} \bnfor \bnfts{$\diffn$} \bnfor \bnfsk
\end{bnf*}
\end{figure}



\begin{example}
Suppose we are interested in those sets which contain element $x$ but not element $y$, or those sets which contain element $y$. We may reduce this query to the following composition. Let
\begin{equation*}
    \Set{A} = \left\{ \Set{Z} \in \Set{U} \colon \Has(\Set{Z}, x) \right\}
\end{equation*}
and
\begin{equation*}
    \Set{B} = \left\{ \Set{Z} \in \Set{U} \colon \Has(\Set{Z}, x) \right\}\,,
\end{equation*}
which are the result of simple membership tests for each set in the universe. Given these two sets, we are interested in the set
\begin{equation*}
    \orfn\!\left(\diffn\!\left(\Set{A}, \Set{B}\right), \Set{B}\right)\,.
\end{equation*}
\end{example}

\subsubsection{Rank-ordered search}
\label{sec:extensions:fuzzyset}
Fuzzy sets are a super-set of  classical (crisp) sets. In a \emph{fuzzy set}, an element has a degree-of-membership between $0$ and $1$. Classical sets are a special case of fuzzy sets where all elements either have $0$ degree-of-membership or $1$ degree-of-membership.

The \emph{approximate map} may be used to implement \emph{approximate fuzzy sets}, where the keys are set elements and the associated values encode degree-of-membership values.

Suppose the values are $p$ bits. Then, let the degree-of-membership be with respect to a function
\begin{equation}
    \operatorname{u} \colon \cisb_P \mapsto [0,1]\,,
\end{equation}
which maps the $P$ bits associated with a particular element $x \in \Sp$ to a degree-of-membership.\footnote{An element $x \notin \Sp$ has $0$ membership in $\Sp$.}
\begin{example}
Let the $p$ bits $b_1, \ldots, b_P$ encode an unsigned integer between $0$ and $2^p-1$. Then,
\begin{equation}
    \operatorname{u}\!\left(b_1, \ldots, b_p\right) = \frac{\sum_{j=1}^{P} 2^{j-1} b_j}{2^P-1}
\end{equation}
maps those $p$ bits to a rational number between $0$ and $1$.
\end{example}

Fuzzy set-theoretic queries operate on degree-of-membership values given by $\operatorname{u}$. Fuzzy operator equivalents to Boolean $\notfn$, $\orfn$, and $\andfn$ are given by
\begin{align}
    \notfn(x) &= 1 - x\,,\\
    \orfn(x, y) &= \max(x, y)\,,\\
    \andfn(x, y) &= \min(x, y)\,,
\end{align}
where $x$ and $y$ are degree-of-membership values.

A hedge function transforms degree-of-membership values as demonstrated by the following example.
\begin{example}
It may be true that an element is a member of a set $\Set{S}$ but it may not be \emph{very} true. If we are interested in sets in which it is \emph{very} true that an element is a member, then the hedge function given by
\begin{equation}
    \very(x) = x^2
\end{equation}
makes the appropriate transformation since $\true\left(\very(x)\right)$ is \emph{false} for larger values of $x$ than $\true(x)$.
\end{example}

A Boolean membership test is eventually needed, in which case defuzzification transforms degree-of-membership values into crisp $\true$ or $\false$ values. For instance,
\begin{equation}
    \true(x) = x \geq K\,,
\end{equation}
where $K \in [0,1]$, transforms degree-of-membership values larger than $K$ to $\true$ and otherwise $\false$. At this point, standard set-theoretic queries as given by BNF~\ref{bnf:set_theoretic} may be used.



In \cite{}, we discussed the fuzzy approximate set. Here, we show how the degree-of-membership may be a function of various other properties having to do with word proximity and word frequency.

Let the degree-of-membership be with respect to a function
\begin{equation}
    \operatorname{u} \colon \cisb \mapsto [0,1]\,,
\end{equation}
which maps the bits associated with a particular element $w \in \Set{W}$ to a degree-of-membership.

Fuzzy set-theoretic queries operate on degree-of-membership values given by $\operatorname{u}$. Fuzzy operator equivalents to Boolean $\notfn$, $\orfn$, and $\andfn$ are given by
\begin{align}
    \notfn(x) &= 1 - x\,,\\
    \orfn(x, y) &= \max(x, y)\,,\\
    \andfn(x, y) &= \min(x, y)\,,
\end{align}
where $x$ and $y$ are degree-of-membership values.

A hedge function transforms degree-of-membership values as demonstrated by the following example.
\begin{example}
It may be true that an element is a member of a set $\Set{S}$ but it may not be \emph{very} true. If we are interested in sets in which it is \emph{very} true that an element is a member, then the hedge function given by
\begin{equation}
    \very(x) = x^2
\end{equation}
makes the appropriate transformation since $\true\left(\very(x)\right)$ is \emph{false} for larger values of $x$ than $\true(x)$.
\end{example}

A Boolean membership test is eventually needed, in which case defuzzification transforms degree-of-membership values into crisp $\true$ or $\false$ values. For instance,
\begin{equation}
    \true(x) = x \geq K\,,
\end{equation}
where $K \in [0,1]$, transforms degree-of-membership values larger than $K$ to $\true$ and otherwise $\false$. At this point, standard set-theoretic queries as given by BNF~\ref{bnf:set_theoretic} may be used.
\end{document}