\documentclass[ ../main.tex]{subfiles}
\providecommand{\mainx}{..}
\begin{document}
\section{The \emph{Typed} Perfect Hash Filter}
The \gls{gls-phf} described in \cref{sec:phf} is an \gls{gls-approxset} over the universe of bit strings $\BitSet^*$. However, injections from a countable set $\Set{X}$ to the set $\BitSet^*$ exist. Let us denote such an injection by
\begin{equation}
    \Encode \colon \Set{X} \mapsto \BitSet^*\,,
\end{equation}
which maps every element $x \in \Set{X}$ to a unique element $y \in \BitSet^*$. Note that in the \gls{gls-phf}, iterating over (or retrieving) elements is not supported and thus \Encode does not need to be decodable.

For instance, if elements of type $\Set{X}$ are a data structure, then \Encode may be its serialization as a sequence of bits.\footnote{The \emph{memory-resident} representation is machine and process-dependent and thus not a suitable representation if the \emph{typed} \gls{gls-phf} is suppose to work across multiple machines or processes.} Alternatively, if the data structure is intended to be efficiently inserted into a \gls{gls-phf}, then something like a simple \emph{hash code} may be used since it does not need to be \emph{decodable}.\footnote{If each element of $\Set{X}$ does not uniquely map to a hash code, then the \gls{gls-fprate} as given by $\fprate$ is only an approximation.}

Consequently, we may map any set $\Set{S} \subset \Set{X}$ to one or more elements in the set $\BitSet^*$ so that the \gls{gls-phf} may approximate it. We denote this the \gls{gls-phf} of type $\Set{X}$. The \gls{gls-phf} of type $\Set{X}$ may approximate any \emph{finite set} $\Set{S}$ where the elements are members of the countable set $\Set{X}$. See \cref{alg:PHF} for an algorithmic description of $\MakeApproxSet$ and \cref{alg:PHF_has} for an algorithmic description of $\Contains$.
\begin{algorithm}[h]
    \caption[Approximate hash set data constructor]{Approximate hash set implementation of $\MakeApproxSet \colon \PS{T} \mapsto \PS{\AT{T}}$.}
    \label{alg:typedPHF}
    \SetKwProg{func}{function}{}{}
    \KwIn
    {
        \begin{itemize}
            \item[$\Set{S}$] The set to approximate where the universe of elements is $\Set{}$.
            \item[$\fprate$] The false positive rate.
            \item[$r$] The load factor.
        \end{itemize}
    }
    \KwOut
    {
        A \emph{typed} \gls{gls-phf} $\PASet{S}(\fprate)$ with a load factor $r$.
    }
    \func{\FuncSty{make\_typed\_approx\_set}{$\Set{S}$, $\fprate$ $\Given$ $r$}}
    {
        $\Set{A} \gets \SetBuilder{ \Encode(x) \in \BitSet^*}{ x \in \Set{S} }$\;
        $\PASet{S}[\BitSet^{*}] \gets$ \MakeApproxSet{$\Set{A}$, $\fprate$ $\Given$ $r$}\;
        \Return $\PASet{S}[\BitSet^{*}]$\;
    }
\end{algorithm}


\begin{algorithm}[h]
    \caption[Approximate hash set contains]{Implementation of $\Contains \colon \PS{\AT{T}} \times T \mapsto \Bool$.}
    \label{alg:typed_PHF_contains}
    \SetKwProg{func}{function}{}{}
    \KwIn
    {
        \begin{itemize}
            \item[$\PASet{X}$] The \gls{gls-phf} of type $\PS{\AT{T}}$ that approximates an objective set $\Set{X}$ of type $\PS{T}$ with a false positive rate $\fprate$.
            \item[$x$] The element of type $T$ to test membership in $\AT{X}$.
        \end{itemize}
    }
    \KwOut
    {
        Returns \True if $x \in \Set{X}$. Otherwise, returns \True with probability $\fprate$ and \False with probability $1 - \fprate$.
    }
    \func{\Contains{$\PASet{X}$, $x$}}
    {
    	\tcp{The approximate hash set is a product type $\BitSet^{n \times m} \times \DataSty{PH}$ where $n$ is a function of the load factor
    	\tcp{$\Encode_T \colon T \mapsto \BitSet^{*}$ is a serializer for $T$.}
        $b \gets \Encode_T(x)$\;
        $q \gets$ \Contains{$\Set{X}_{\BitSet^{*}}$, $b$}\;
        \Return $q$\;
    }
\end{algorithm}

\subsection{C++ implementation}
Stuff for C++ goes here.
\end{document}