\documentclass[ ../main.tex]{subfiles}
\providecommand{\mainx}{..}
\begin{document}
\section{A C++ implementation}
\label{sec:impl}
Random positive approximate sets are a well-defined concept. Any data structure $T$ and set of functions dependent upon $T$ required by the concept is an implementation of the model.
Any generic algorithm (generic programming) parameterized by $R$ which assumes $R$ models a random approximate set may be applied to the data structure $T$.

We show a simple implementation.

\subsection{Subtypes: dynamic polymorphism}
TODO: break up each function/class/maybe method into separate listings and then
talk about each separately.

Open set of data types, closed set of operations.

We may also use \emph{dynamic polymorphism}. A dynamic polymorphic C++ 
interface of the \emph{random approximate set} abstract data type is detailed next.

%\begin{listing}
%\inputminted[breaklines,frame=lines,linenos]{c++}{code/random_approximate_sets_adt/include/approximate_set/dynamic%/aset.hpp}
%\caption{Dynamic polymorphic C++ interface for approximate sets.}
%\label{lst:aset}
%\end{listing}

Assuming the elements of an approximate set are \emph{non-enumerable}, i.e., only membership tests may be performed using the member-of $\SetContains \colon \Set{U} \times \PowerSet(\Set{U})$ interface, set-theoretic operations may be implemented by storing each approximate set \emph{as-is} and composing them together. For example, if we are given two approximate sets $\ASet{S}[1]$ and $\ASet{S}[2]$, the union $\ASet{S}[1] \cup \ASet{S}[2]$ is implemented by performing membership tests on both $\ASet{S}[1]$ and $\ASet{S}[2]$ and returning \True if either membership test returns \True, i.e.,
\begin{equation}
\SetContains[x][\left(\SetUnion[\ASet{S}[1]][\ASet{S}[2]]\right)] \equiv \left(\SetContains[x][\ASet{S}[1]]\right) \lor \left(\SetContains[x][\ASet{S}[2]]\right)\,.
\end{equation}

Any set-theoretic composition may be implemented as a combination of unions and complements as described in \cref{sec:set_theory}. A C++ implementation for taking the union of two approximate sets is given by \cref{lst:asetunion} and a C++ implementation for taking the complement of an approximate set is given by \cref{lst:asetcomp}.

% code\random_approximate_sets_adt\include\approximate_set\dynamic

The following C++ implementation of a Perfect Hash Filter of type $X$ models a first-order rate-distorted set $\AT{\RV{R}}[\fprate][\fnrate]$ of type $X$ where $\fprate = 2^{-K}$ and $\fnrate$ is the product of $\fprate$ and the possible rate-distortion of the (rate-distorted) perfect hash function.

%\begin{minted}[escapeinside=||,mathescape=true]{c++}
%template <
%	typename X = string_view,
%	typename Y = array<HashType>,
%	int K, // $\fprate = 2^{-K}$
%	// Models $X \mapsto$ Integer
%	template <typename> typename PerfectHashFn,
%	// Models $X \mapsto$ Integer
%	typename <typename> typename HashFn,
%	typename <typename> typename HashContainer,
%	bool Complement = false
%>
%class perfect_hash_filter
%{
%public:
%	static_assert(K >= 0, "K must be a non-negative number");
%	
%	using HashType = decltype(HashFn<X>{}(X()));
%
%	// Set models with values convertible to X
%	template <typename Set>
%	perfect_hash_filter(
%		Set A,   // A is objective set to approximate
%		float r, // r is load factor
%		HashFn<X> h = HashFn<X>{}) : h_(h)
%	{
%		auto N = ceil(size(A) / r);
%		ph_ = PerfectHashFn<X>(h, A, r);
%		hs_.resize(N);
%		for (auto x : A)
%			hs_[ph_(x)] = h_(x) % K;
%	{
%	
%	auto contains(X x) const
%	{
%		return hs_[ph_(x)] == h_(x) % K;
%	}
%
%	auto fnr() const
%	{
%		return rate_distortion(ph_) * (1 - fpr());
%	}
%
%	auto fpr() const
%	{
%		return power_probability<K,2>{}; // 2^{-K}
%	}
%
%private:
%	PerfectHashFn<X,HashFn> ph_;
%	HashFn<X> h_;
%	Container<HashType> hs_;
%};
%\end{minted}



\begin{minted}[escapeinside=||,mathescape=true]{c++}
template <
	// Expected false positive rate $\fprate = 2^{-K}$.
	int K,
	// Models a function of the type $string\_view \mapsto int$.
	template PerfectHashFn,
	bool Complement = false
>
class perfect_hash_filter
{
public:
	template <typename Set>
	perfect_hash_filter(
		Set A,   // A is objective set to approximate
		float r, // r is load factor
		uint32_t seed)
{
	h = make_hash_fn(seed);
	ph_ = build_perfect_hash_fn(h_, A, r);

	auto N = ceil(size(A) / r);	
	hs_.resize(N);
	for (auto x : A)
	hs_[ph_(seed, x)] = h_(x) % K;
{

auto contains(X x) const
{
	return hs_[ph_(x)] == h_(x) % K;
}

private:
	PerfectHashFn<hash_fn> ph_;
	hash_fn h_;
	vector<uint32_t> hs_;
};
\end{minted}

\begin{minted}{c++}
template <int K, typename PerfectHashFn>
auto fpr(perfect_hash_filter<K, PerfectHashFn, false>)
{
	return power<K,2>{}; // 2^{-K}
}
\end{minted}

\begin{minted}{c++}
template <int K, typename PerfectHashFn>
auto fnr(perfect_hash_filter<K, PerfectHashFn, false> A)
{
	return rate_distortion(A.hash_fn())*(1-fpr(A));
}
\end{minted}

\begin{minted}{c++}
template <int K, typename PerfectHashFn>
auto fpr(perfect_hash_filter<K, PerfectHashFn, true>)
{
	return rate_distortion(A.hash_fn())*(1-fpr(A)); ???
}
\end{minted}

\begin{minted}{c++}
template <int K, typename PerfectHashFn>
auto fnr(perfect_hash_filter<K, PerfectHashFn, true>) { return power<-K,2>{}; }
\end{minted}

The complement unary operator is given next.
\begin{minted}{c++}
template <typename K, bool Complement>
perfect_hash_filter<K,!Complement> operator~(perfect_hash_filter<K,Complement> a)
{
	return perfect_hash_filter<K,!Complement>(a);
}
\end{minted}

The perfect hash filter, as a \emph{first-order} rate-distorted set, is closed under complements, i.e., it remains a first-order approximation after complementation.



%template <unsigned int K2>
%bool operator==(
%perfect_hash_filter<X2, PerfHashFn, K2, HashFn, CoderFn> const & rhs) const
%{
%	if (h_ != rhs.h_ ||
%	ph_ != rhs.ph_ ||
%	coder_ == rhs.coder_)
%	{
%		return false;
%	}
%	
%	for (size_t i = 0; i < static_cast<size_t>(N_); ++i)
%	{
%		if (hashes_[i] != rhs.hashes_[i])
%		return false;
%	}
%	return true;            
%}
%
%
%template <
%typename T,
%typename PerfHashFn,
%unsigned int K1,
%unsigned int K2,
%typename HashFn,
%typename CoderFn,
%bool Complement
%>
%constexpr bool subset_eq(
%perf_hash_filter<T,PerfHashFn,K1,HashFn,CoderFn,Complement> const lhs,
%perf_hash_filter<T,PerfHashFn,K2,HashFn,CoderFn,Complement> const rhs)
%{
%	return lhs == rhs;
%}
%
%template <
%typename T1,
%typename T2,
%typename PerfHashFn1,
%typename PerfHashFn2,
%unsigned int K1,
%unsigned int K2,
%typename HashFn1,
%typename HashFn2,
%typename CoderFn1,
%typename CoderFn2,
%bool Complement1,
%bool Complement2
%>
%constexpr bool subset(
%perf_hash_filter<T1,PerfHashFn1,K1,HashFn1,CoderFn1,Complement1> const lhs,
%perf_hash_filter<T2,PerfHashFn2,K2,HashFn2,CoderFn2,Complement2> const rhs)
%{
%	return false;
%}
%
%template <
%typename T,
%typename PerfectHashFn,
%int K,
%typename HashFn,
%template <typename> typename Container,
%bool Complement
%>
%bool is_member(
%T lhs,
%perfect_hash_filter<T,PerfectHashFn,K,HashFn,Container,Complement> const rhs)
%{
%	return rhs.contains(lhs);
%}
%
%template <
%typename T,
%typename PerfHashFn,
%unsigned int K,
%typename HashFn,
%typename CoderFn,
%bool Complement
%>
%constexpr bool contains(
%perf_hash_filter<T,PerfHashFn,K,HashFn,CoderFn,Complement> const lhs,
%T rhs)
%{
%	return lhs.contains(rhs);
%}
%}



%\begin{listing}
%	%\inputminted[breaklines]{c++}{code/random_approximate_sets_adt/include/approximate_set/dynamic/aset_complement.hpp%}
%	\caption{C++ implementation of the complement of approximate sets.}
%	\label{lst:asetcomp}
%\end{listing}


If Container models the concept of a PackedContainer, Container<Integer> requires $\mathcal{O}(m/r log2 N)$ bits.



\end{document}