\documentclass[ ../main.tex]{subfiles}
\providecommand{\mainx}{..}
\begin{document}
\section{A C++ implementation}
\label{sec:impl}
Random positive approximate sets are a well-defined concept. Any data structure $T$ and set of functions dependent upon $T$ required by the concept is an implementation of the model.
Any generic algorithm (generic programming) parameterized by $R$ which assumes $R$ models a random approximate set may be applied to the data structure $T$.

We show a simple implementation.

\subsection{Dynamic polymorphism}
TODO: break up each function/class/maybe method into separate listings and then
talk about each separately.

We may also use \emph{dynamic polymorphism}. A dynamic polymorphic C++ 
interface of the \emph{random approximate set} abstract data type is detailed next.

%\begin{listing}
\inputminted[breaklines,frame=lines,linenos]{c++}{code/random_approximate_sets_adt/include/approximate_set/dynamic/aset.hpp}
%\caption{Dynamic polymorphic C++ interface for approximate sets.}
%\label{lst:aset}
%\end{listing}

Assuming the elements of an approximate set are \emph{non-enumerable}, i.e., only membership tests may be performed using the member-of $\SetContains \colon \Set{U} \times \PowerSet(\Set{U})$ interface, set-theoretic operations may be implemented by storing each approximate set \emph{as-is} and composing them together. For example, if we are given two approximate sets $\ASet{S}[1]$ and $\ASet{S}[2]$, the union $\ASet{S}[1] \cup \ASet{S}[2]$ is implemented by performing membership tests on both $\ASet{S}[1]$ and $\ASet{S}[2]$ and returning \True if either membership test returns \True, i.e.,
\begin{equation}
\SetContains[x][\left(\SetUnion[\ASet{S}[1]][\ASet{S}[2]]\right)] \equiv \left(\SetContains[x][\ASet{S}[1]]\right) \lor \left(\SetContains[x][\ASet{S}[2]]\right)\,.
\end{equation}

Any set-theoretic composition may be implemented as a combination of unions and complements as described in \cref{sec:set_theory}. A C++ implementation for taking the union of two approximate sets is given by \cref{lst:asetunion} and a C++ implementation for taking the complement of an approximate set is given by \cref{lst:asetcomp}.

% code\random_approximate_sets_adt\include\approximate_set\dynamic

\begin{listing}
	\inputminted[breaklines]{c++}{code/random_approximate_sets_adt/include/approximate_set/dynamic/aset_union.hpp}
	\caption{C++ implementation of the union of approximate sets.}
	\label{lst:asetunion}
\end{listing}

\begin{listing}
	\inputminted[breaklines]{c++}{code/random_approximate_sets_adt/include/approximate_set/dynamic/aset_complement.hpp}
	\caption{C++ implementation of the complement of approximate sets.}
	\label{lst:asetcomp}
\end{listing}

Other compositions may be implemented by composing set-union and set-complement, e.g., set-intersection is given by
\begin{minted}{c++}
auto make_intersection(auto s1, auto s2) =
{
    return make_complement(make_union(
        make_complement(s1),
        make_complement(s2)));
}
\end{minted}

\end{document}